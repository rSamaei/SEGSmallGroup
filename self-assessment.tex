%%% This is a LaTeX template for the small group project self-assessment document.
%%% To compile this document, make sure pdflatex is installed on your workstation and use:
%%% pdflatex self-assessment.tex in a UNIX CLI.
\documentclass[11pt,a4paper]{article}
\usepackage[margin=2.5cm]{geometry}
\renewcommand{\familydefault}{\sfdefault}

\begin{document}
\title{Small Group Project Self-Assessment}
\author{}
\date{}
\maketitle

%%% Lines starting with %%% are comment lines.
%%% Fill out the form by removing all comment lines and replacing them by the suggested content.

\noindent\textbf{Team name}: %%% Insert team name here

\noindent\textbf{URL of the deployed application}: %%% Insert URL of the deployed application here

\noindent\textbf{Team members}:
\begin{itemize}
\setlength\itemsep{0em}
\item Mohammed Shafayat Hossain
\item Muhammad Mushtaq Ahmad
\item Endrin Hoti
\item Adnan Hussain
\item Reza Samaei
\end{itemize}

\noindent\textbf{If the team agrees on an effective team size, specify it here.  Otherwise, state that the team does not agree.} 
%%% Insert effective team size or �did not agree� here.

\section*{Submission checklist}
\begin{tabular}{|p{15mm}|p{13cm}|}
\hline
Check & Task \\
\hline
& The work that is submitted has been produced by members of the team only, without input from other students or third parties.  All use of examples from the internet and generative AI are declared in the README.md file.\\
\hline
& The team is familiar with the required deliverables for this project, outlined in the small group project handbook, available on KEATS.\\
\hline
& The team's shared GitHub repository has been registered on Team Feedback.  All commits are attributed to a member of the team.  The shared repository has a default branch that corresponds to the accepted version of the software.\\
\hline
& The version of the web application that the team wishes to submit has been deployed.  The deployed system is seeded with data that meets the requirements outlined in the specifications.  The system is accessible using the required access credentials.  All functionality has been tested.\\
\hline
& All team members listed above agree with the claims made in this checklist.\\
\hline
\end{tabular}

\section*{Objective}
Briefly outline the business objective of the system you have built.  Aim to identify what the system you developed aims to achieve (do not regurgitate the overall assignment).  This section should consist of a single paragraph.  Aim for less than 100 words.

%%% Insert your objective description here.

\section*{Features}
Identify the main features of your team's application.  Identify what type of user can access this (or their username), and any other constraints.  If necessary, describe how/where the feature should be accessed.  Small or very obvious features need not be included.  The examiners will be using this to assess and find the functionality available to users of the application.  

\begin{tabular}{|p{50mm}|p{40mm}|p{50mm}|}
\hline
Feature: Calendar & Any type of user - @johndoe, @janedoe, @charlie for admin, tutor, student respectively & After logging in, students and tutors can see a mini Calendar on their dashboard, then clicking on the bottom beneath it will redirect you to the full calendar. All users also have a direct link to the calendar on the navbar. Only admins on the calendar page have access to the search bar, where you can search for users, subjects or proficiencies. \\
\hline
Feature: Match Requests & Admin users - @johndoe & After loggin in as the admin, you can see a card in the dashboard with the title 'Unmatched Requests', clicking on 'View all requests' will get you to the page for matching student requests with tutors who are able to teach it. (You are also able get here through the navbar) Upon reaching this page, you will see all requests, if a session was requested within two weeks of the term starting, it will be marked as late. You are able to quick match by selecting on the drop down menu, or for more details you can click on 'View Details', which will take you to the highlighted page. On the Requested Session Page you can also search for sessions by the username, subject or proficiency.\\
 \hline
Feature: Tutor subjects - Allow tutors to submit what subject they can teach as well as remove the subject and modify their proficiency.
 \hline
Feature: View all users - Admins can view/see all users in a table and choose what users to delete as well as see user information
 \hline
Feature: search bar - All users can use the search bar at the top of the page to filter out requests and records on the database
 \hline
Feature: Modify/delete requests - Allow students to modify and delete any requests
 \hline
Feature: Pending requests - All users can see their pending requests and admins can see all pending requersts. Tutors can accept or reject pending requests
 \hline
 Feature: Tutors are able to view and modify the subjects they are able to teach as well as their proficiency in them. The table shows their ID, name, subject, proficiency, and an action coloumn, which allows the tutor to delete their subject or update their proficiency level. @janedoe is an example of a tutor who can access this feature by clicking on the 'View All Subjects' on the dashboard. 
 \hline
 Feature: All types of users are able to view the days which were requested on both the pending approvals and matched requests boxes on the dashboard.
 \hline
  & & \\
 \hline
  & & \\
 \hline
\end{tabular}
%%% Add rows as needed.

\end{document}